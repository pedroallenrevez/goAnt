%
%  $Description: Author guidelines and sample document in
%  LaTeX 2.09$ 
%
%  $Author: ienne $ $Date: 1995/09/15 15:20:59 $ $Revision:
%  1.4 $
%

\documentclass[times, 10pt,twocolumn]{article}
\usepackage{latex8} \usepackage{times}
\usepackage{graphicx}
\usepackage{amsmath}
\usepackage{mathtools}



%\documentstyle[times,art10,twocolumn,latex8]{article}

\newcommand{\forceindent}{\leavevmode{\parindent=1em\indent}}

%-------------------------------------------------------------------------
%take the % away on next line to produce the final
%camera-ready version 
\pagestyle{empty}

%------------------------------------------------------------------------- 
\begin{document}

\title{\ GoAnt\\ Ant Colony Optimization Algorithm implementation in Golang}

			 \author{Miguel Vera, Pedro Lopes\\
				 Utrecht University\\ Multi-Agent Systems \\
                 Computer Science Department\\  
                 \{miguel.coimbra.vera,
				 pedroallenrevez\}@gmail.com }

\maketitle \thispagestyle{empty}

\begin{abstract} 
		Routing and path finding are major problems in the realm of Computer Science. One
	of the several existing algorithms used to solve this is Ant Colony Optimization(ACO). We 
	developed a version of this algorithm in Golang and a separate accompanying visualizer.
	This separation allows the ACO algorithm to be used to solve real world problems and 
	not be bound by the limits of a simulation environment. In the following report we 
	analyze the positive aspects and limitations of the algorithm and our own implementation.
    This report was written for the Multi-Agent Systems Course in Utrecht University.
\end{abstract}

%------------------------------------------------------------------------- 
\Section{Introduction}
		In the world of Computer Science, we tend to imagine this field as something entirely
	new and because of that, tend to devise ways of doing things from the ground up. More 
	times than we realize, a lot of these problems have already been solved in one way or
	another in Nature. Routing and path finding are a good example of this, in the natural 
	world this problem has been solved by ants. Ants are limited animals, they do not possess
	eyes nor any other mechanisms for long range orientation like ultra-sounds, they only
	have *TODO*. With these limitations (and due to the laws of evolution (TODO, is right term?)) they 
	evolved in a way that utilizes these limited sensing abilities in an optimal way.
	
	    It is easy to understand the parallels that can be drawn between an ant and an agent.    Both,
	
%------------------------------------------------------------------------- 
\Section{Ant Colony Optimization}

	

%------------------------------------------------------------------------- 
\SubSection{Natural Optimization Algorithm}
Being inspired by the behaviour of ants on a naturally occuring process, the Ant Colony Optimization algorithm is a probablistic thechnique used in solving computational problems that can be reduced to finding good paths through grahps, and we classified it as a natural optimization algorithm. The ants wander aimlessly around until they find food, returning to their colony leaving pheromone trails behind. The other ants can detect the pheromones laid on the ground and can follow it to find food themselves. This property is called \textbf{Stygmergy}, and is what allows this algorithm to be fit for a Multi Agent System environment. Agents interact through the environment without needing direct communication, and requiring no planning. The more ants a colony has, the more attractive the paths are going to become - this \textbf{Autocatalytic} emerging propoerty allows the set of agents to scale realy well towards finding a solution in the shortest distance possible. The comparison between ammount of energy spent and food reward, drawn from a real life scenario, provides a metaheurisitic optimization solution towards finding the shortest path in various types of environments and problems.
	
	 
%------------------------------------------------------------------------- 
\SubSection{Ants}

The ant worker agents exhibit two behaviours during the algorithm, \textbf{forward} mode and \textbf{backward} mode. 


In forward mode the ants evaluate all neighbouring nodes (exlcluding the last visited node) and calculate the probability to traverse a node according to the ammount of pheromone laid on the path
% insert formula here
where tau is the ammount of pheromone laid between x and y, and eta is the desirability of the path, usually being a distance function 1/distance\_{x,y}, defining the \textit{a priori} attractiveness of the path. Alpha and beta are both experimental values used to control the influence of the two factors.


When the ant finds food, it enter backward mode. It starts retracing its steps back to the colony leaving a pheromone trail behind. The ants do this multiple times until they converge into a solution.
%pseudo codigo das ants

%------------------------------------------------------------------------- 
\SubSection{Problems and optimizations}
With the immense diversity of graphs, sometimes ants enter a loop, because the pheromone becomes stronger and stronger, using its positive feedback in a bad way making our agents stuck, even worse when retracing its steps back, doing the same loop again and laying dangerous pheromones that could derail our other agents. To tackle this problem a loop deletion optimization technique was used. When an ant finds food, before returning to the colony, the route travelled by the ant is checked for loops, and eliminates them. This produces a much more natural flow to the agents.

%insert pheromone pseudo
Over time some trails may also become predominant due to the effect of pheromones. This can lead to a problem called \textbf{missed opportunity} where the ants may just miss a path that is shorter because it isn't that attractive. To solve this a global pheromone decay was used. When all the ants finish their tours, the pheromone value of each node is udpated according to 
% insert formula here
where ro is the evaporation coefficient and delta tau is the ammount of pheromone deposited by the kth ant.



%------------------------------------------------------------------------- 
\SubSection{Applications of the Algorithm}
	The Ant Colony Algorithm belongs 
	
%------------------------------------------------------------------------- 
\Section{Visualizer}

%------------------------------------------------------------------------- 
\SubSection{Overview}

%------------------------------------------------------------------------- 
\SubSection{Interaction with ACO}

%------------------------------------------------------------------------- 
\Section{Optimizations}
	Loop Deletion

%------------------------------------------------------------------------- 
\Section{Conclusion}


%------------------------------------------------------------------------- 
%\Section{Style Guidelines}
%
%Please follow the steps outlined below when submitting your manuscript to
%the IEEE Computer Society Press. Note there have been some changes to the
%measurements from previous instructions. 
%
%%------------------------------------------------------------------------- 
%\SubSection{Margins and page numbering}
%
%All printed material, including text, illustrations, and charts, must be
%kept within a print area 6-7/8 inches (17.5 cm) wide by 8-7/8 inches
%(22.54 cm) high.  Do not write or print anything outside the print area.
%Number your pages lightly, in pencil, on the upper right-hand corners of
%the BACKS of the pages (for example, 1/10, 2/10, or 1 of 10, 2 of 10, and
%so forth). Please do not write on the fronts of the pages, nor on the
%lower halves of the backs of the pages.
%
%
%%------------------------------------------------------------------------ 
%\SubSection{Formatting your paper}
%
%All text must be in a two-column format. The total allowable width of the
%text area is 6-7/8 inches (17.5 cm) wide by 8-7/8 inches (22.54 cm) high.
%Columns are to be 3-1/4 inches (8.25 cm) wide, with a 5/16 inch (0.8 cm)
%space between them.  The main title (on the first page) should begin 1.0
%inch (2.54 cm) from the top edge of the page. The second and following
%pages should begin 1.0 inch (2.54 cm) from the top edge. On all pages, the
%bottom margin should be 1-1/8 inches (2.86 cm) from the bottom edge of the
%page for $8.5 \times 11$-inch paper; for A4 paper, approximately 1-5/8
%inches (4.13 cm) from the bottom edge of the page.
%
%%------------------------------------------------------------------------- 
%\SubSection{Type-style and fonts}
%
%Wherever Times is specified, Times Roman may also be used.  If neither is
%available on your word processor, please use the font closest in
%appearance to Times that you have access to.
%
%MAIN TITLE. Center the title 1-3/8 inches (3.49 cm) from the top edge of
%the first page. The title should be in Times 14-point, boldface type.
%Capitalize the first letter of nouns, pronouns, verbs, adjectives, and
%adverbs; do not capitalize articles, coordinate conjunctions, or
%prepositions (unless the title begins with such a word).  Leave two blank
%lines after the title.
%
%AUTHOR NAME(s) and AFFILIATION(s) are to be centered beneath the title and
%printed in Times 12-point, non-boldface type.  This information is to be
%followed by two blank lines.
%
%The ABSTRACT and MAIN TEXT are to be in a two-column format. 
%
%MAIN TEXT. Type main text in 10-point Times, single-spaced.  Do NOT use
%double-spacing. All paragraphs should be indented 1 pica (approx. 1/6 inch
%or 0.422 cm). Make sure your text is fully justified---that is, flush left
%and flush right.  Please do not place any additional blank lines between
%paragraphs.  Figure and table captions should be 10-point Helvetica
%boldface type as in \begin{figure}[h] \caption{Example of caption.}
%\end{figure}
%
%\noindent Long captions should be set as in \begin{figure}[h]
%	\caption{Example of long caption requiring more than one line. It is not
%	typed centered but aligned on both sides and indented with an additional
%margin on both sides of 1~pica.} \end{figure}
%
%\noindent Callouts should be 9-point Helvetica, non-boldface type.
%Initially capitalize only the first word of section titles and first-,
%second-, and third-order headings.
%
%FIRST-ORDER HEADINGS. (For example, {\large \bf 1.  Introduction}) should
%be Times 12-point boldface, initially capitalized, flush left, with one
%blank line before, and one blank line after.
%
%SECOND-ORDER HEADINGS. (For example, {\elvbf 1.1. Database elements})
%should be Times 11-point boldface, initially capitalized, flush left, with
%one blank line before, and one after. If you require a third-order heading
%(we discourage it), use 10-point Times, boldface, initially capitalized,
%flush left, preceded by one blank line, followed by a period and your text
%on the same line.
%
%%------------------------------------------------------------------------- 
%\SubSection{Footnotes}
%
%Please use footnotes sparingly% \footnote {% Or, better still, try to
%	avoid footnotes altogether.  To help your readers, avoid using footnotes
%	altogether and include necessary peripheral observations in the text
%	(within parentheses, if you prefer, as in this sentence). and place them
%	at the bottom of the column on the page on which they are referenced.
%	Use Times 8-point type, single-spaced.
%
%
%%------------------------------------------------------------------------- 
%\SubSection{References}
%
%List and number all bibliographical references in 9-point Times,
%single-spaced, at the end of your paper. When referenced in the text,
%enclose the citation number in square brackets, for example~\cite{ex1}.
%Where appropriate, include the name(s) of editors of referenced books.
%
%%------------------------------------------------------------------------- 
%\SubSection{Illustrations, graphs, and photographs}
%
%All graphics should be centered. Your artwork must be in place in the
%article (preferably printed as part of the text rather than pasted up).
%If you are using photographs and are able to have halftones made at a
%print shop, use a 100- or 110-line screen. If you must use plain photos,
%they must be pasted onto your manuscript. Use rubber cement to affix the
%images in place. Black and white, clear, glossy-finish photos are
%preferable to color. Supply the best quality photographs and illustrations
%possible. Penciled lines and very fine lines do not reproduce well.
%Remember, the quality of the book cannot be better than the originals
%provided. Do NOT use tape on your pages!
%
%%------------------------------------------------------------------------- 
%\SubSection{Color}
%
%The use of color on interior pages (that is, pages other than the cover)
%is prohibitively expensive. We publish interior pages in color only when
%it is specifically requested and budgeted for by the conference
%organizers. DO NOT SUBMIT COLOR IMAGES IN YOUR PAPERS UNLESS SPECIFICALLY
%INSTRUCTED TO DO SO.
%
%%------------------------------------------------------------------------- 
%\SubSection{Symbols}
%
%If your word processor or typewriter cannot produce Greek letters,
%mathematical symbols, or other graphical elements, please use
%pressure-sensitive (self-adhesive) rub-on symbols or letters (available in
%most stationery stores, art stores, or graphics shops).
%
%%------------------------------------------------------------------------ 
%\SubSection{Copyright forms}
%
%You must include your signed IEEE copyright release form when you submit
%your finished paper. We MUST have this form before your paper can be
%published in the proceedings.
%
%%------------------------------------------------------------------------- 
%\SubSection{Conclusions}
%
%Please direct any questions to the production editor in charge of these
%proceedings at the IEEE Computer Society Press: Phone (714) 821-8380, or
%Fax (714) 761-1784.
%
%%------------------------------------------------------------------------- 
%\nocite{ex1,ex2} \bibliographystyle{latex8} \bibliography{latex8}

\end{document}
